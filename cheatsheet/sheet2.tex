\documentclass[10pt,landscape]{article}
\usepackage{multicol}
\usepackage{calc}
\usepackage{ifthen}
\usepackage[landscape]{geometry}


% This sets page margins to .5 inch if using letter paper, and to 1cm
% if using A4 paper. (This probably isn't strictly necessary.)
% If using another size paper, use default 1cm margins.
\ifthenelse{\lengthtest { \paperwidth = 11in}}
	{ \geometry{top=.5in,left=.5in,right=.5in,bottom=.5in} }
	{\ifthenelse{ \lengthtest{ \paperwidth = 297mm}}
		{\geometry{top=1cm,left=1cm,right=1cm,bottom=1cm} }
		{\geometry{top=1cm,left=1cm,right=1cm,bottom=1cm} }
	}

% Turn off header and footer
\pagestyle{empty}
 

% Redefine section commands to use less space
\makeatletter
\renewcommand{\section}{\@startsection{section}{1}{0mm}%
                                {-1ex plus -.5ex minus -.2ex}%
                                {0.5ex plus .2ex}%x
                                {\normalfont\large\bfseries}}
\renewcommand{\subsection}{\@startsection{subsection}{2}{0mm}%
                                {-1explus -.5ex minus -.2ex}%
                                {0.5ex plus .2ex}%
                                {\normalfont\normalsize\bfseries}}
\renewcommand{\subsubsection}{\@startsection{subsubsection}{3}{0mm}%
                                {-1ex plus -.5ex minus -.2ex}%
                                {1ex plus .2ex}%
                                {\normalfont\small\bfseries}}
\makeatother

% Define BibTeX command
\def\BibTeX{{\rm B\kern-.05em{\sc i\kern-.025em b}\kern-.08em
    T\kern-.1667em\lower.7ex\hbox{E}\kern-.125emX}}

% Don't print section numbers
\setcounter{secnumdepth}{0}


\setlength{\parindent}{0pt}
\setlength{\parskip}{0pt plus 0.5ex}


% -----------------------------------------------------------------------

\begin{document}

\raggedright
\footnotesize
\begin{multicols}{3}


% multicol parameters
% These lengths are set only within the two main columns
%\setlength{\columnseprule}{0.25pt}
\setlength{\premulticols}{1pt}
\setlength{\postmulticols}{1pt}
\setlength{\multicolsep}{1pt}
\setlength{\columnsep}{2pt}

\begin{center}
     \Large{\textbf{ICAT \LaTeX\ CheatSheet}} \\
\end{center}

\section{Document classes}
\begin{tabular}{@{}ll@{}}
\verb!book!    & Default is two-sided. \\
\verb!article! & No \verb!\part! or \verb!\chapter! divisions. \\
\end{tabular}

Used at the very beginning of a document:
\verb!\documentclass{!\textit{class}\verb!}!.  Use
\verb!\begin{document}! to start contents and \verb!\end{document}! to
end the document.


\subsection{Common \texttt{documentclass} options}
\newlength{\MyLen}
\settowidth{\MyLen}{\texttt{letterpaper}/\texttt{a4paper} \ }
\begin{tabular}{@{}p{\the\MyLen}%
                @{}p{\linewidth-\the\MyLen}@{}}
\texttt{10pt}/\texttt{11pt}/\texttt{12pt} & Font size. \\
\end{tabular}

Usage:
\verb!\documentclass[!\textit{opt,opt}\verb!]{!\textit{class}\verb!}!.


\subsection{Packages}
\settowidth{\MyLen}{\texttt{multicol} }
\begin{tabular}{@{}p{\the\MyLen}%
                @{}p{\linewidth-\the\MyLen}@{}}
%\begin{tabular}{@{}ll@{}}
\texttt{url}       & Insert URL: \verb!\url{!%
                        \textit{http://\ldots}%
                        \verb!}!.
\end{tabular}

Use before \verb!\begin{document}!. 
Usage: \verb!\usepackage{!\textit{package}\verb!}!


\subsection{Title}
\settowidth{\MyLen}{\texttt{.author.text.} }
\begin{tabular}{@{}p{\the\MyLen}%
                @{}p{\linewidth-\the\MyLen}@{}}
\verb!\author{!\textit{text}\verb!}! & Author of document. \\
\verb!\title{!\textit{text}\verb!}!  & Title of document. \\
\verb!\date{!\textit{text}\verb!}!   & Date. \\
\end{tabular}

These commands go before \verb!\begin{document}!.  The declaration
\verb!\maketitle! goes at the top of the document.



\section{Document structure}
\begin{multicols}{2}
\verb!\part{!\textit{title}\verb!}!  \\
\verb!\chapter{!\textit{title}\verb!}!  \\
\verb!\section{!\textit{title}\verb!}!  \\
\verb!\subsection{!\textit{title}\verb!}!  \\
\end{multicols}

%---------------------------------------------------------------------------

\section{Text properties}

\subsection{Font face}
\newcommand{\FontCmd}[3]{\PBS\verb!\#1{!\textit{text}\verb!}!  \> %
                         \verb!{\#2 !\textit{text}\verb!}! \> %
                         \#1{#3}}
\begin{tabular}{@{}l@{}l@{}l@{}}
\textit{Command} & \textit{Declaration} & \textit{Effect} \\
\verb!\textbf{!\textit{text}\verb!}!                    & %
        \verb!{\bfseries !\textit{text}\verb!}!               & %
        \textbf{Bold series} \\
\end{tabular}

The command (t\textit{tt}t) form handles spacing better than the
declaration (t{\itshape tt}t) form.

\subsection{Font size}
\setlength{\columnsep}{14pt} % Need to move columns apart a little
\begin{multicols}{2}
\begin{tabbing}
\verb!\footnotesize!          \= \kill
\verb!\tiny!                  \>  \tiny{tiny} \\
\verb!\normalsize!            \>  \normalsize{normalsize} \\
\end{tabbing}
\end{multicols}
\setlength{\columnsep}{1pt} % Set column separation back

These are declarations and should be used in the form
\verb!{\small! \ldots\verb!}!, or without braces to affect the entire
document.


\section{Text-mode symbols}

\subsection{Line and page breaks}
\settowidth{\MyLen}{\texttt{.pagebreak} }
\begin{tabular}{@{}p{\the\MyLen}%
                @{}p{\linewidth-\the\MyLen}@{}}
\verb!\\!          &  Begin new line without new paragraph.  \\
\end{tabular}


\subsection{Miscellaneous}
\settowidth{\MyLen}{\texttt{.rule.w..h.} }
\begin{tabular}{@{}p{\the\MyLen}%
                @{}p{\linewidth-\the\MyLen}@{}}
\verb!\today!  &  \today. \\
\end{tabular}


\section{Sample \LaTeX\ document}
\begin{verbatim}
\documentclass[11pt]{book}
\usepackage{url}
\author{Name}
\title{Template}
\date{1. August 2010}
\begin{document}
\maketitle
\chapter{my first section}
\section{section}
Sectiontext
\subsection{subsection}
%comment
Normal text \\ %textbreak
\tiny           % set font to tiny
tiny testtext \\
\normalsize     % set it back to normal
But you can easily embed {\tiny tiny text} into other sizes.
\subsection{Link}
Link to ICAT: \\
\url{http://www.icat.ac.id/}
\chapter{Timechapter}
\section{Whats the date?}
\today
\end{document}
\end{verbatim}



\end{multicols}
\end{document}
