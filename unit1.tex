%\documentclass[draft,hyperref={pdfpagelabels=false}]{beamer}
\documentclass[handout]{beamer}
\let\Tiny=\tiny
\hypersetup{pdfpagemode=FullScreen}
\usepackage[ngerman]{babel}
\usepackage[utf8]{inputenc}
\usepackage{graphics}
\usepackage{listings}
\usepackage{verbatim}
%\setbeamertemplate{navigation symbols}{}

\usetheme{Boadilla}

\usecolortheme{beaver}
\usefonttheme{professionalfonts}
\useinnertheme{rounded}
\useoutertheme{smoothbars}
%\useoutertheme{sidebar}

\definecolor{lGray}{gray}{.90}

\newcommand{\code}[1]{\colorbox{lGray}{\texttt{#1}}}
\author{Christian Kniep}

\begin{document}
\title{LaTeX and Git}  
\institute[ICAT Bandung]{Internation Center of Applied Technologies Bandung}
\date[\today]{\today} 

\begin{frame}
	\titlepage
\end{frame} 

\begin{frame}
	\frametitle{Table of content}
	\tableofcontents
\end{frame} 


\section{LaTeX-Introduction} 
	\subsection{History}
		\begin{frame}{Once upon a time}
			\begin{itemize}
				\item<1-> Donald  E. Knuth (born 1938) is a computer scientist and Professor Emeritus of the Art of Computer Programming at Stanford University.
                \item<2-> In 1969 he wrote the book 'The Art Of Computer Programming'. He is the Godfather of a hugh amount of cumputer related stuff.
                \item<3-> The book was typeseted with metal letters; the good old style.
                \item<4-> In 1976 he wants to typeset the 2nd edition, but the typesetting changed to a digital way. His favorite font was not available and the digital print was awful.
            \end{itemize}
		\end{frame}
        \begin{frame}{Lamport TeX}
			\begin{itemize}
                \item<1-> So he decided to make it by himself and he created \TeX, with the goals
                \begin{itemize}
                    \item<2-> allow anybody to produce high-quality books using a reasonable amount of effort
                    \item<3-> provide a system that would give the exact same results on all computers, now and in the future
                \end{itemize}
				\item<4-> based on \TeX Leslie Lamport creates a dialect which includes popular Macros, so that it is easier to use than the original TeX.
            \end{itemize}
		\end{frame}
    \subsection{Control Sequences}
        \begin{frame}[fragile]{basic}
			\begin{itemize}
				\item<1-> Control-sequences are commands to format: \\
                    \verb!\command[optional parameter]{parameter}!
                \item<2-> e.g. the documentclass definition with 11pt font as book: \\
                    \verb!\documentclass[11pt]{book}!
                \item<3-> another useful sequence is the start-sequence \verb!\begin{something}!
                          and its twin \verb!\end{something}! \\
                          All text between them will be influrenced by 'something'
            \end{itemize}
		\end{frame}
    \subsection{Create a document}
        \begin{frame}[fragile]{header}
			\begin{itemize}
                \item<1-> To create a document we have to create a minimal header
                \item<2-> \textbf{Document-setting} \\
                will define the global environment of the document
                \begin{itemize}
                    \item<2-> \verb!\documentclass[options]{<class>}! 
                \end{itemize}
                \item<3-> \textbf{Packages} \\
                    are used to provide addition functionality within the document (syntax-highlighting, url-links, etc.)
                \begin{itemize}
                    \item<3-> \verb!\usagepackage{<name>}! 
                \end{itemize}
                \item<4-> \textbf{Title} \\
                    To have some variables defined which could be used within the layout at least this should be set
                \begin{itemize}
                    \item<4-> \verb!\author{<name>}! 
                    \item<4-> \verb!\title{<title>}! 
                    \item<4-> \verb!\institute{<date>}! 
                    \item<4-> \verb!\date{<date>}! 
                \end{itemize}
            \end{itemize}
		\end{frame}
        \begin{frame}[fragile]{body}
			\begin{itemize}
                \item<1-> \verb!\begin{document}! marks the start, \verb!\end{document}! marks the end                 \begin{itemize}
                \item<2-> after that there is usualy a Titlepage: \verb!\maketitle!
                \item<3-> maybe you want the table of content? \verb!\tableofcontents!
            \end{itemize}
		\end{frame}
        

\end{document}
